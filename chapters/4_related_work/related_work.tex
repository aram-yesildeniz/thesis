\chapter{Related Research}
\label{chapter:related_work}


% [Last chapter]

% [Research Question]

% [This chapter: Why its here. Describe shortly all sections from this chapter]

% general research
% research about metrics
% research about tools

% [Next chapter]


% ----------------------------------------------------------------------


\section{General Research in the Field of Web Performance}


% 2014 Singal: WEB ANALYTICS: STATE-OF-ART & LITERATURE ASSESSMENT
%\paragraph{2014 Singal}

%I.
%- Describes history of web analytics and tools
%- Provides definitions and taxonomy for metrics
%- Describes log file vs page tagging
%- Describes KPIs

%II.
%- Lit. overview for KPIs and Web Metrics
%- Lit. overview for "Trust"
%- Lit. overview for "Fuzzy"
%-> What are does categories?


%III.
%- Some other literature worth mentioning


%IV.
%- Describes 8 open challenges for researchers


%\paragraph{2015 Bekavac}

%- Two parts:
%	- 1: Some general overview of web analytics, tools and metrics, KPIs etc
%	- 2: Empirical study about employees satisfaction of used web analytics tools

%1: 
%- 9 web business models and 5 common goals
%- Hypothesis: Web analytics tools track and improve a user’s satisfaction with web-based business models.
%- Web anayltics defintion. Log files vs Site Tagging
%- Web Analytics process
%- Tools: 5 categories, Process of selecting tool, Table with features of different tools
%- Web metrics categories, Table with business models and their KPIs

%2:
%- Which tools are used for which purpose / Activity
%- Users satisfaction



% ----------------------------------------------------------------------


\section{Research on Web Performance Tools}



% https://www.kaushik.net/avinash/web-analytics-tool-selection-three-questions-to-ask-yourself/
%\paragraph{Kaushik 2007}
%- Provides 3 questions which help to choose web analytics tools



% 2011 Nakatani: A web analytics tool selection method: an analytical hierarchy process approach

%- Gives some arguments why web analytics is important for business
%- Provides different categorizations for web analytics tools
%- Gives pros and cons of log file analysis and page tagging
%- Provides tool selection method based on AHP (Analytic Hierarchy Process)

%"Web analytics tools collect click-stream data, track users navigation paths, process and present the data as meaningful information.
%- Categorizations:
%1: By 4 different data collection methods
%2: SaaS vs in-house
%3: mobile vs non-mobile
%4: Time lag




% 2013 Wang: WProf: Profiler / new tool for finding bottlenecks
% - PLT metric
% - How page loading works, CRP, dependencies
% - They create WProf as a tool to analyse dependencies and performance bottlenecks




% 2016 Kaur: Tool comparison
% This paper evaluated the university websites of Punjab using four automated tools and gives the comparative results of various factors using these tools.



% 2009 Jansen ch. 6.3
% - paraphrases 10 tips by Kaushik... i would neet to cite them directly


% Croll 2009 Choosing a analytics platform p. 144
%- free vs paid
%- real time vs long term
%- hosted vs in-house
%- data portability


% 2011 Marek: Choose a program


% 2015 Bekavac --> Move this to related work ??
% - 5 categories of tools
% - Process of selecting tool
% - Table with features of tools


% 2016 Bartuskova: 1.2 Services for Website Speed Testing



% 2019 Kumar
% - free / open source tools
% - proprietary tools
%- Service Hosted Software
%- GA most popular one


% 2020 Quintel: Matomo --> put this to related work ??
% Lighthouse Performance Scoring ?




% Tools: Kessler 2016 p. 576 ??




% ----------------------------------------------------------------------


\section{Research on Web Performance Metrics}


% Categorizations as seen in section X.


% 2018 Netravali: Ready Index: new metric




% 2019 Enghard Pitfalls



% ----------------------------------------------------------------------



%TODO Which category ??

% 2015  Cito: IDENTIFYING WEB PERFORMANCE DEGRADATIONS THROUGH SYNTHETIC AND REAL-USER MONITORING

% 2016 Bartuskova: Our results indicate that the choice of service and location affects significantly results of website speed testing.


