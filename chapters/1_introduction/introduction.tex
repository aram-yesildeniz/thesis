\chapter{Introduction}

\begin{itemize}
	\item Explain structure of this thesis and maybe give some "eckdaten"
	\item This chapter: Describe shortly all sections from this chapter
	\item In the next chapter...
\end{itemize}


% -----------------------------------
% -----------------------------------


\section{E-Commerce}


% [Internet]
\subsection{The Internet}

In the last 50 years, a new technology emerged, spread over the entire world and influenced many aspect of most peoples life.
Within the turmoil of the cold war, the United State's Advanced Research Projects Agency (ARPA) established in 1957 a communication network to bring together universities and their researches all around the country in order to be able compete against the USSR.
% cite 2011 COhen
What started as a tool for scientific collaboration evolved a half century later into the internet, a global network and phenomenon, to which every user with a dedicated device has access and can contribute to.
The internet is an integral part, if not the backbone of today's everyday life.
Users of the internet use it for everything, that is sending emails, watching television, chatting with friends, 
order lunch, checking the weather for the next day or renting motorized scooters.

% [Numbers]

In 2021, the internet has 4.66 billion users, which is around 60\% of the world population.\footnote{Following statistics are taken from \url{https://datareportal.com/reports/digital-2021-germany} [14.05.2021]}
Compared to 2020, the number of internet users increased by 7.3\%.
In Europe, more than 90\% of the population are also internet users.
For a developed country like Germany, the numbers are even more impressing:
94\% of the German population are using the internet with an average daily time of over 5h.

The numbers demonstrate impressive that the internet is an integral part of our daily life.
Along the rise of internet users, transactions and processes falling under the term of e-commerce rise.
Before discussing the term "e-commerce" and take a grasp at its history and types, some statistics are presented to demonstrate the importance of e-commerce.


% [E-Commerce]
\subsection{E-Commerce}

\subsubsection{Introduction}

From the global data report, we can see that over 90\% of the world population visited an online retail site.
Over 76\% of the world population purchased a product online.
For most categories growth is over 15\%.
Again, for a western country like Germany, the numbers are higher:
92.5\% of the german population visited an online retail site and over 80\% purchased a product online.
And the usage is growing: growth of amount spent in category food and personal care is 28.6\%, and 17.6\% for fashion and beauty.

Revenue in e-commerce is constantly growing over the last 20 years\footnote{\url{https://einzelhandel.de/presse/zahlenfaktengrafiken/861-online-handel/1889-e-commerce-umsaetze} [14.05.2021]}, topping to 57.8 billion in 2019.

% [Corona: Even more growth]

The COVID-19 pandemic with its implications had and still has an not negligible impact on the growth of e-commerce.
Several measures were taken to stop the spread of the virus and the number of deaths, one of which was to minimize physical interaction between people.
This leads consequently to a shift of human interactions to the internet.
Along this, e-commerce benefits.
Bhatti et al concludes that "e-commerce enhanced by COVID-19.". %cite 2020 bhatti

% [E-Commerce History]
\subsubsection{Short History}

E-Commerce, or electronic commerce, is according to the \textit{Encyclopædia Britannica} about "maintaining relationships and conducting business transactions that include selling information, services, and goods by means of computer telecommunications networks."\footnote{\url{https://www.britannica.com/technology/e-commerce} [19.05.2021]}
In short, e-commerce is about buying and selling products and services via the internet.

%cite hermogeno and check for Tian when paper available:
The success of e-commerce is tightly coupled to the vast advances of internet technology in the past years. 
Worth mentioning is the development of the Electronic Data Interchange (EDI) starting from the 1960s, which standardised the communication between two machines.

Personal computers in 1980s and one of the first examples of online shopping is from CompuServe who introduced Electronic Mall in 1984.

Another milestone is Word Wide Web, introduced in 1990 by Tim Berners-Lee, made internet accessible to everyone.
First browser by Tim.

Social media since 2000 again offeres new possibilities for businesses and consumers alike to participate in e-commerce, e.g. for marketing, selling channels

New devices such as smart phones and tablets again decreases the hurdle to participate in e-commerce.
While in the time dimension e-commerce was already available all the time, with the new mobility its also available everywhere.
More flexible.

With the ongoing progress in technology, also e-commerce can expect a shining future with trends such as AI recommendation systems, outstanding UX thanks to Virtual Reality, simple payment methods with crypto etc.\footnote{\url{https://www.spiralytics.com/blog/past-present-future-ecommerce/} [19.05.2021]}


% [Types]
\subsubsection{Types}

Multiple types in e-commerce are existing. They reflect the possible combinations between the actors business, customer and government. %cite dos santos

\begin{itemize}
\item B2B: Business to Business
\item B2C: Business to Consumer
\item C2C: Consumer to Consumer
\item G2C: Government to Consumer
\item B2G: Business to Government
\item G2G: Government to Government
\end{itemize}

% [B2C]
\paragraph{B2C}
% TODO: [Small Info]
Online shop, basically is a normal shop online.
The service includes Product presentation, Order and purchase process, Payment and Delivery.

From an users perspective who wants to become a business, multiple solutions exists, e.g. Shopify, ePages, Magento or WooCommerce. % cite Handbuch Online-Shop

% [Amazon]
A famous example of an B2C company is Amazon.
On the 16th of July in 1995, Amazon launched as a website.%cite Der Allesverkäufer p. 47
Entered the stock market on the 15th of May 1997, started with a price of X, today is Y.\footnote{\url{https://finance.yahoo.com/quote/AMZN?p=AMZN} [19.05.2021]}
Today Amazon employs over 1 million employees.\footnote{\url{https://www.statista.com/statistics/234488/number-of-amazon-employees/} [19.05.2021]}



% [C2C]
\paragraph{C2C}
This is...
One of the most famous examples is eBay.



% [Pro and Con]
\subsubsection{Assessment}

Advantage is, that No need of a real shop house. The shopping room is virtual as a website
Online 24 7

Part of growing market
Scalable
Targeting
Data analysis
Not much money needed to start
Multi-Channel

% hermogeno:
wide variety of products
lower costs
less time consuming..
personalized customer experience, etc.

Cons: Speed of market
The downside is that there is no direct concat with customer.
Naturally this means that the overall user experience needs to be very good.
Performance is one part of ux


% erfolgreicher online handel for dummies


% [Performance of online shop is important]

Quintesenz: The performance of an online shop is important.
For now we leave the term performance as a idea about the speed of the online shop, e.g. how long it takes to load the page, and how the user perceives this performance.
Later we will see that measuring performance is not that trivial and a lot of ideas and metrics are existing to measure it.




% ------------------------------------------


\subsection{User Satisfaction and Performance}

\begin{itemize}
\item What is user satisfaction and why is it important
\item Prove that Page Speed = Money
\item Introduction to A/B Testing. How and for which reasons A/B Testing is used in practice.
\end{itemize}


% Why it matters / performance = money:
% 2016 Witt

% code.talks 2019 - Mobile Site Speed and the Impact on E-Commerce
% Felix Gessert and Wolfram Wingerath. Mobile Site Speed and the Impact on E-Commerce
% more PPs here...


\paragraph{A/B Testing}

% How to quantify, measure impact of performance?
% Try to be scientifically solid

% 2012 Kessler AB Tests
% 2016 Kohavi
% 2017 Kohavi
% 2018 Morys
% 2018 Fabijan
% 2020 Heinemann 4.1.4





% ------------------------------------------
% ------------------------------------------



\section{Web Analytics}

\begin{itemize}
\item Historical background and contextualisation, usage, definition
\item Web Analytics Process
\item Mechanisms, Measurement methods / Collecting data: Log file analysis, client site page tagging, alternatives
\item KPIs ?
\end{itemize}



% Definitions:
% -----------


% 2011 Nakatani:
% Web analytics is used to understand online customers and their behaviors, design actions influential to them, and ultimately foster behaviors beneficial to the business and achieve the organization’s goal.


% 2014 Singal: "Web Analytics is the objective tracking, collection, measurement, reporting and analysis of quantitative internet data to optimize websites and web marketing initiatives."

% 2015 Bekavac:

% "According to the official definition of [13], web analytics refers to a combination of (a) measuring, (b) acquisition, (c) analyzing and (d) reporting of data collected from the Internet with the aim of understanding and optimizing web experience."

% "the analysis of qualitative and quantitative data on the website in order to continuously improve the online experience of visitors, which leads to more efficient and effective realization of the company’s planned goals"




% History:
% -------


% 2009 Croll p. 69
% 2009 Jansen ch. 3

% 2014 Singal
% - Genesis of Web Analytics (nice graphic) with WAA, DAA
% - History of tools

% 2015 Zheng
% 2015 Bekavac
% 2019 Hussaina
% 2019 Kumar


% Process:
% --------


% 2009 Jansen ch. 6.2
% 2009 Waisberg
% 2012 Kumar

% 2015 Bekavac:
% - Copies from Waisberg and Kaushik


% 2019 Hussaina
% 2019 Kumar


% Log vs Page tagging:
% --------------------


% 2009 Waisberg

% Hassler ch. 2

% Wolle Draft

% 2011 Marek


% 2011 Nakatani: Pros and cons


% 2014 Singal provides table with pros and cons

% 2015 Bekavac

% 2019 Kumar



% KPIs:
% -----

% 2014 Singal describes KPIs for different businesses

% 2015 Bekavac:
% - KPI definition as part of web analytics process
% - Table with KPIs related to business model


\subsection{Web Performance}

\begin{itemize}
\item What is web performance? Why it matters
\item Overview of factors that impact performance, bottlenecks
\item Overview of measurement methods, techniques and metrics
\end{itemize}



% Bottlenecks of performance, big picture:
% 2013 Grigorik: Latency as bottleneck and not bandwith
% 2016 Witt




\subsection{Tools}

% TODO: move this to related work ?

\begin{itemize}
\item Some short overview about existing tools
\item Conclude that I use WPT for synthetic performance testing and GA for RUM
\end{itemize}

% 2011 Marek: Choose a program



% 2011 Nakatani:
% "Web analytics tools collect click-stream data, track users navigation paths, process and present the data as meaningful information.
% - Categorizations:
% 1: By 4 different data collection methods
% 2: SaaS vs in-house
% 3: mobile vs non-mobile
% 4: Time lag



% 2015 Bekavac
% - 5 categories of tools
% - Process of selecting tool
% - Table with features of tools


% 2016 Bartuskova: 1.2 Services for Website Speed Testing
% 2016 Kaur: Tools for Measuring the Performance of Websites
% 2019 Hussaina
% 2019 Kumar

% 2020 Quintel: Matomo
% Lighthouse Performance Scoring

% 2020 Heinemann 4.1.4




% -----------------------------------
% -----------------------------------



\section{Research Question}

\begin{itemize}
\item Difficulty of defining scope
\item Measuring performance of a web site impacts its performance or other effects take place / Observer effect
\item Why the research question is relevant
\end{itemize}


% Tools: Kessler 2016 p. 576

% Measuring Real User Performance in the Browser: Avoiding the Observer Effect























