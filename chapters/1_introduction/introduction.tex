\chapter{Introduction}

[tbd]

\begin{itemize}
	\item Explain structure and main goal of this thesis
	\item Describe shortly all sections from this chapter and what the reader can expect
	\item Give short outlook to following chapter
\end{itemize}


%TODO
% internet schreibweise: Gross oder kleines I
% when to use italic for a term and when not ?
% numbers: 1´000 or 1,000 or 1000 ? 12.3 or 12,3 ?
% for example: e.g., i.e., "for example" .. ?
% website schreibweise: "web site" or "website"
% schreibweise Front End Frontend, Backend Back End...
%- Trennung Konzept und Implementation
%- Grafiken: In Caption direkt Referenz (taken from ...)
% Use PDFs (from svg) for graphics!

%- Kapitel Überschriften noch griffiger

%- Trennung Web analytics Web Performance Metriken klarer 


% ROTER FADEN ganz klar hervorheben: Jeweils vor und nach jedem Kapitel und Abschnitt beschreiben was war davor, was danach, warum steht das hier und wie ist die logik vom text.

% For each chapter, try to make some kind of image, diagram, table or other abstraction with a quick text to summarize most important stuff -> wrap up

% Use same tables everywhere:  longtable and standard table


% ---------------------------------------------------------------------------------------------------------------------------------
% ---------------------------------------------------------------------------------------------------------------------------------


\section{Research Question}
\label{chapter:research_question}
%TODO ausführlicher, evtl link to other chapters ?

The e-commerce industry is booming and there are no signs that this trend is reversing; on the other hand.
Performance plays an important role in terms of customer satisfaction and how this directly affects business revenue.
To better understand e-commerce website visitors, page tagging is widely used and implemented.

Several questions and issues can arise in this area and context, such as: Does page tagging affect the website's performance?
Intuitively, it can be said that loading additional JavaScript will reduce the performance of the website, depending on parameters such as the script size and network condition.
But are there more unpredictable side effects?
Do the various techniques of embedding a tracking script affect the data collected and measured?
Will the various tracking scripts supplied interfere with each other?

A hypothesis of this work is that tracking tools slow down the monitored websites, reduce the speed and performance of the website and thus have an unfavourable effect on the user experience.

These questions are to be investigated within the scope of this thesis.


% ------------------------------------------------


\section{Goal of this Thesis}

This thesis has several goals:

The Internet and websites in general are complicated, complex, and tangled.
Although basic HTML structures are standardized, each website follows its own form and is unique and sui generis.
In order to conduct a controlled experiment and test hypotheses, one goal is to approximate real websites with an artificial, laboratory-generated website that is completely controlled and manipulated by the researcher.

The aim is to create a reliable, but also convincing test environment in order to model and reproduce real behaviour.

Once the test environment is up and running, performance measurement issues need to be addressed.
The aim is to measure, collect, visualize and analyse performance data.

As we will see in chapter X, there are many metrics for measuring performance.
Another goal of this work is to establish something like a taxonomy of performance metrics.





% The research motivation is to investigate if, and how tracking tools are slowing down websites and which particular features of those tools impact how and why the websites performance.

% Within this investigation, the trade-off between “meaningful” collected data and the websites performance is of special interest. The assumption is that slim, performant tracking tools may not aggregate that much useful data as heavy and presumably the website decelerating tools.



% ------------------------------------------------


\section{Chapter Outline}

[tbd]

% TODO move this to beginning of introduction ?

Chapter 1 was about...
In Chapter 2 we see,
Chatper 3...













