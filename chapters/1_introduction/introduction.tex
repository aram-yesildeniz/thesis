\chapter{Introduction}

\begin{itemize}
	\item Explain structure and main goal of this thesis
	\item Introduction Chapter: Describe shortly all sections from this chapter and what the reader can expect
	\item Give short outlook to following chapter
\end{itemize}


% -----------------------------------
% -----------------------------------


\section{E-Commerce}


% [Internet]
\subsection{The Internet}

In the last 50 years, a new technology emerged, spread over the entire world and influenced many aspect of most peoples life.
Within the turmoil of the cold war, the United State's Advanced Research Projects Agency (ARPA) established in 1957 a communication network to bring together universities and their researches all around the country in order to be able compete against the USSR (\cite{2011Cohen}). % cite 2011 COhen
What started as a tool for scientific collaboration evolved a half century later into the internet, a global network and phenomenon, to which every user with a dedicated device has access and can contribute to.
The internet is an integral part, if not the backbone of today's everyday life.
Users of the internet use it for everything, that is sending emails, watching television, chatting with friends, 
order lunch, checking the weather for the next day or renting motorized scooters.

% [Numbers]

In 2021, the internet has 4.66 billion users, which is around 60\% of the world population.\footnote{Following statistics are taken from \url{https://datareportal.com/reports/digital-2021-germany} [14.05.2021]}
Compared to 2020, the number of internet users increased by 7.3\%.
In Europe, more than 90\% of the population are also internet users.
For a developed country like Germany, the numbers are even more impressing:
94\% of the German population are using the internet with an average daily time of over 5h.

Those numbers demonstrate impressive that the internet is an integral part of our daily life.
Along the rise of internet users, transactions and processes falling under the term of e-commerce rise.
Before discussing the term "e-commerce" and take a grasp at its history and types, some statistics are presented to demonstrate the importance of e-commerce.


% [E-Commerce]
\subsection{E-Commerce}

\subsubsection{Introduction}

From the global data report, we can see that over 90\% of the world population visited an online retail site.
Over 76\% of the world population purchased a product online.
For most categories growth is over 15\%.
Again, for a western country like Germany, the numbers are higher:
92.5\% of the german population visited an online retail site and over 80\% purchased a product online.
And the usage is growing: growth of amount spent in category food and personal care is 28.6\%, and 17.6\% for fashion and beauty.

Revenue in e-commerce is constantly growing over the last 20 years, topping to 57.8 billion in 2019.\footnote{\url{https://einzelhandel.de/presse/zahlenfaktengrafiken/861-online-handel/1889-e-commerce-umsaetze} [14.05.2021]}

% [Corona: Even more growth]

The COVID-19 pandemic with its implications had and still has an not negligible impact on the growth of e-commerce.
Several measures were taken to stop the spread of the virus and the number of deaths, one of which was to minimize physical interaction between people.
This leads consequently to a shift of human interactions to the internet.
Along this, e-commerce benefits.
Bhatti et al. (\cite{2020Bhatti}) conclude that "e-commerce enhanced by COVID-19".

% [E-Commerce History]
\subsubsection{Short History}

E-Commerce, or electronic commerce, is according to the \textit{Encyclopædia Britannica} about "maintaining relationships and conducting business transactions that include selling information, services, and goods by means of computer telecommunications networks."\footnote{\url{https://www.britannica.com/technology/e-commerce} [19.05.2021]}
In short, e-commerce is about buying and selling products and services via the internet.

% TODO check for Tian when paper available:
The success of e-commerce is tightly coupled to the vast advances of internet technology in the past years, like for example the development of the Electronic Data Interchange (EDI) starting from the 1960s, which standardised the communication between two machines.

Personal computers in 1980s and one of the first examples of online shopping is from CompuServe who introduced Electronic Mall in 1984.

Another milestone is Word Wide Web, introduced in 1990 by Tim Berners-Lee, made internet accessible to everyone.
First browser by Tim.

Social media since 2000 again offeres new possibilities for businesses and consumers alike to participate in e-commerce, e.g. for marketing, selling channels

New devices such as smart phones and tablets again decreases the hurdle to participate in e-commerce.
While in the time dimension e-commerce was already available all the time, with the new mobility its also available everywhere.
More flexible.
\cite{2019Hermogeno}.

With the ongoing progress in technology, also e-commerce can expect a shining future with trends such as AI recommendation systems, outstanding UX thanks to Virtual Reality, simple payment methods with crypto etc.\footnote{\url{https://www.spiralytics.com/blog/past-present-future-ecommerce/} [19.05.2021]}


% [Types]
\subsubsection{Types}

Multiple types in e-commerce are existing. They emerge from the possible combinations between the actors \textit{business}, \textit{consumer} and \textit{government} (\cite{2017DosSantos}).

\begin{itemize}
\item B2B: Business to Business
\item B2C: Business to Consumer
\item C2C: Consumer to Consumer
\item G2C: Government to Consumer
\item B2G: Business to Government
\item G2G: Government to Government
\end{itemize}

% [B2C]
\paragraph{B2C}
Business to Consumer in e-commerce describes basically online shopping, by means of a business offering its services and products to the consumer over the World Wide Web.
The consumer can browse within an online shop through the presented products and services and order them directly via the website.
A variety of payment and delivery options conclude the B2C type (\cite{2020Heinemann}).
% cite Heinemann with correct page p. 75 ?

For a aspiring business, multiple ready to use software solutions to install an online shop are existing, for example. Shopify, ePages, Magento or WooCommerce (\cite{2019Steireif}).

% [Amazon]
A famous example of a B2C company is Amazon.
On the 16th of July in 1995, Amazon launched as a website and entered the stock market on the 15th of May 1997 (\cite{2019Stone}). % cite also that this is directly from page p. 47 ?
Amazon is successful, the stock started with a price of X, which is at the time of this writing at Y.\footnote{\url{https://finance.yahoo.com/quote/AMZN?p=AMZN} [19.05.2021]}
Today Amazon employs over 1 million employees\footnote{\url{https://www.statista.com/statistics/234488/number-of-amazon-employees/} [19.05.2021]} and serves the wishes of 200 million paying prime members.\footnote{\url{https://www.statista.com/statistics/829113/number-of-paying-amazon-prime-members/} [20.05.2021]}


% [Pro and Con]
By taking a quick look at the pros and cons of maintaining an online shop, we can see that some of the advantages are that there is no need of a real house to present and sell the products, the virtual shop is available to the consumer at any time and has to closing hours; there is a high potential for the online shop as it is part of growing market; online business is scalable; due to tracking algorithms precise targeting as well as data analysis is possible; to start with an online business, not that much float is needed and there are in general lower costs; it is possible to provide a personalized customer experience; 

Some disadvantages are that the speed of market is rapid, competitors arise everyday everywhere, technology evolves quickly while consumers expectations go high (\cite{2019Hermogeno}, \cite{2020Lang}).


% [Performance of online shop is important]
Another downside is that there is no direct or physical concat with the consumer.
As described above, online shopping takes place in the World Wide Web domain.
Consequently, in person interaction between a buyer and a seller is not possible and the shopping event takes place on a website.
Deriving from this fact, the overall virtual user experience needs to be outstanding in order to stay ahead of competition.
The performance of the online shop is one part of the user experience.

In the next section, I will describe the findings between the correlation between user satisfaction and the performance of the retailers web presence.



% ------------------------------------------


\subsection{User Satisfaction and Performance}

% [User Satisfaction]
The aim of this thesis is not to deep dive into terms and concepts or the non-trivial problem of defining user satisfaction, usability or the like.
Therefore the term user satisfaction is in this context loosely defined as how happy the user is with the website he or she interacts with.\footnote{For a discussion cf. "User satisfaction measurement" in } %cite 2010 Islam

% [Performance]
For this context, performance can be understood as the the speed of an online shop, e.g. how long it takes the page to load, how quickly the user can interact with the page, and how the user perceives the performance of the website.
Later we will see that measuring performance is not that trivial and a lot of ideas and metrics are existing to measure it.


% [SpeedHub]
% TODO: directly cite studies from powerpoint
\subsubsection{SpeedHub}

A plethora of information and studies about the phenomenon of user satisfaction and web site performance is collected at \textit{SpeedHub.org}, a portal by \textit{Baqend} in cooperation with \textit{Google} which provides "the largest systematic study of Mobile Site Speed and the Impact on E-Commerce."\footnote{\url{https://www.speedhub.org/} [21.05.2021]}
On the hub, not only studies and reports are available, but also collections of videos and blog posts.

% [code.talks 2019]
In his presentation at code talks 2019, Felix Gessert summarizes the findings and provides insights to the most relevant aspects and questions of the study: %cite 2019 Gessert

% [User Profile and Psychology]
The first observation when tackling the question regarding a correlation between the performance of a system and the user satisfaction, is that the users have to be distinguished which leads to the concept of a \textit{User Profile}: Regarding gender, young woman are the most demanding consumers and buy less on slow pages.
Generally, people between 18 and 24 have higher expectations regarding site speed than their older counterparts.

There are also differences between nations and regions, for example people from Japan have the highest expectations, which for a certainty coheres with the technological progress in this country.
Not only the expectations themselves differ geographically, but also how speed influences the users, for example "speed influences New Yorkers more than Californians."

% cite 
% Think Fast, The 2019 Page Speed Report Stats & Trends for Marketers, Unbounce, 2019
% Brain Food, Speed Matters, Designing for Mobile Performance, Google, 2017


What all users have in common is their human psychology. With respect to performance, researchers generally suggest to keep waiting times under 1000 ms in order to keep the users attention.
% cite
% I. Girgorik, High performance browser networking, O'Reilly Media 2013
% Jakob Nielsen, Usability Engineering, Morgan Kaufmann, 1994

% [Devices]
After considering the user itself, the next step is to investigate the influence of the device in use: Studies show that mobile users are more likely to buy products and services than their colleagues using a desktop computer, where iOS users have generally more expectations regarding site speed.
% cite
% Android vs iOS market share 2019 Q2. DeviceAtlas, 2019.
 
% [Context]
Last but not least is the context and the users condition important, where naturally relaxed and calm users perceive sites faster than stressed users or people that are in a hurry.
Also when on the move, users experience sites slower.
% Performance Matters. 9 Key Consumer Insights, Akamai, 2014.


% [Studies]
There are many real world examples and studies existing which prove and demonstrate the importance of site speed with respect to user satisfaction and eventually revenue:
\textit{Amazon} fount out that a decrease of 100 ms in page loading leads to -1\% conversion rates.
If the site loads 100 ms faster, \textit{Walmart} observed that the revenue increases by 1\%.
For \textit{Zalando}, increasing site speed by 100 ms leads to an uplift of 0.7\% revenue per session.
% cite
% Greg Linden. Make Data Useful. Standford Data Mining Class CS345A, 2006.
% Shuhei Kagawa, Jeff Cybulski, David Martin Jones, et al.. Landing Time Matters. Zalando Tech Blog, 2018
% C. Crocker, A. Kulick, B. Ram. Real-User Monitoring at Walmart. SF & SV Web Performance Group, 2012

% [SEO]
Search Engine Optimization is heavily impacted by load speed:
For \textit{Google}, 500 ms slower sites lead to a decrease of 20\% in traffic.
\textit{GQs} traffic increased by 80\% after the page load went down from 7 s to 2 s.
And for \textit{Pinterest}, 40\% faster loads led to 15\% more SEO traffic.
% cite
% Lucia Moses. How GQ Cut Its Webpage Load Time By 80 Percent. Digiday, 2015
% Marissa Mayer, Conference Keynote, Web 2.0, 2006.
% Sam Meder, Vadim Antonov, Jeff Chang. Driving User Growth With Performance Improvements, Pinterest Blog, 2017



% [Engagement & Satisfaction]
Also the user engagement and satisfaction rely heavily on load times:  \textit{Forrester} noted an increase of 60\% for the session length while brining down the load time by 80\%.
\textit{Akamai} monitored that the bounce rate climbed up incredible 103\% when the load time increased by 2 seconds.
And for the \textit{AberdeedGroup}, the customer satisfaction dropped by 16\% at one more second delay in response times.
% cite
% Forrester. The Total Economic Impact Of Accelerated Mobile Pages, 2017.
% Akamai, Akamai Online Retail Performance Report: Milliseconds Are Critical, Akamai Blog, 2017
% The Performance of Web Applications: Customers Are Won or Lost in One Second. Aberdeen Group, 2008.



To summarize, many studies and real world examples prove and demonstrate that faster web sites and online shops cause a better user experience and typically lead to happier customers. 
Concluding in commercial terms, one can say with a certainty that page speed equals money.
\\

In order to properly test the impact of performance on the users, a scientific method is needed.
A/B testing as a controlled experiment is one of them and will be explained in the next section.
After a discussion of A/B Testing, I will move on to the examination of \textit{Web Analytics}, a term which subsumes methods, tools and instruments for businesses to better understand their business and customers.


\subsubsection{A/B Testing}

% 2012 Kessler 17.2  AB Tests ?


% 2016 Kohavi
Controlled experiments such as A/B Testing are not a new tool for scientists and researchers and have been used already in the 1920s. % cite Kohavi 2016
With the rise of the internet in the 1990s, the concept has been adopted to the online domain and is as of today widely used by big companies such as Amazon, Facebook or Google to directly test ideas and hypothesises on a live system.
Controlled experiments such as A/B Testing are utilised to support decision making and to deliver "causal relationship with high probability". % cite Kohavi 2016
They enable a data driven and quantitative validation of the hypothesis. % cite Morys 2018

Controlled experiments help to test hypothesis and questions of the form: "If I change feature X, will it help to improve the key performance indicator Y?"

To answer this question,  two systems are needed: Version A, the control variant or default version, and a slightly different version B, called the treatment.
If more than two versions or one treatment should be evaluated at the same time,  an A/B/n split test has to be implemented.
For an univariable setup, only one variable differs between the systems, where in a multivariable structure, more than one variables are changed at the same time.

Usually, the users of the system are randomly split into two groups and testing is directly performed with real users on a production system.
Beneficial, also when comparing with other experimental setups, is, that the users and participants are not aware that they are part of an experiment, which leads to less bias and side effects.
In order to measure the differences and the user behaviour, web analytics has to be integrated within the system.
% cite Kohavi 2016


% 2018 Morys
%- 5 steps:
%- 1. quality of hypothesis according to SOR Paradigma
%- 2. quality of testconcept: isolation and contrast of changed parameters
%- 3. quality of implementation and quality assurance: running tests should not be visible by user (e.g. suddenly UI changes) and performance of website should not be impacted
%- 4. quality of measurement: significant difference in primary goals
%- 5. quality of statistical interpretation: amount of data, statistical methods


A short and general discussion about controlled experiments in computer science is in chapter X. %link chapter about controlled experiments


% [Transition to Web Analytics]

To resume with the question of performance and user satisfaction,  A/B testing enables to serve two different versions of the same site to two groups, one site being slow, the other one fast, at the same time without the users knowing.
With web analytics implemented, it is possible to measure how the different systems and user groups behave.

What web analytics exactly is, what tools are available and how a web analytics process looks like, will be discussed in the next section.




% Heinemann 2020 A/B Tests nach 4.1.4



% ------------------------------------------------------------------------------------
% ------------------------------------------------------------------------------------
% ------------------------------------------------------------------------------------
% ------------------------------------------------------------------------------------






\section{Web Analytics}

First some definitions, then quickly summarize the history, and then technical aspects of collecting data.


\subsection{Introduction}

% [Definitions]
What is Web Analytics?
Going through the literature, makes it clear that multiple definitions are existing:

Nakatani et al. state that "Web analytics is used to understand online customers and their behaviors, design actions influential to them, and ultimately foster behaviors beneficial to the business and achieve the organization's goal." % cite 2011 Nakatani
According to this definition, web analytics is about getting insights of the users using the system, not only who or what they are, but also how they interact with the system.
Additionally, the definition stressed that the underlying motivation of web analytics is are the achievement business goals.

Singal et al provide a more technical definition by pointing out that "Web Analytics is the objective tracking, collection, measurement, reporting and analysis of quantitative internet data to optimize websites and web marketing initiatives." % cite 2014 Singal with note that this definition is directly taken from Kaushik
Again, the ultimate target is to drive business forward, but backed up with data science methods and instruments such as tracking, collecting and analysis of vast amount of data.

Bekavac et al provide a similar definition by pointing out that web analytics is "the analysis of qualitative and quantitative data on the website in order to continuously improve the online experience of visitors, which leads to more efficient and effective realization of the company's planned goals." %cite 2015 Bekavac:

% [Use Cases]
% 2015 Zheng
- 4 usages:
- Improving website/application design and user experience
- Optimizing e-Commerce and improving e-CRM on customer orientation, acquisition and retention
- Tracking and measuring success of actions and programs such as commercial campaigns
- Identifying problems and improving performance of web applications

Summarizing above definitions, we can see that web analytics is composed of two important aspects: A data driven, information focused and technical aspect of collecting and analysis data about the users, and a commercial perspective, which provides the main motivation of collecting the data beforehand, by setting business goals.


% 2019 Kumar
- 2 categories: on-site and offsite



\subsection{Short History}

TODO check out Wolles dissertation!

% 2009 Croll p. 69
- Logfiles, 1996 ELF
- GetStats, one of the first tools to present log files nicely to user
- Only interesting for IT in the beginning
- With 1.5 billion internet users in 2009, also interesting for marketing
- Focus shifted to visitor, so that his behaviour can be linked e.g. with purchase.  Cookies helped with this
- Segmentation as crucial step
- First web analytics companies tracked log files, which needs it heavy infrastructure
- Generally the shift was from technical aspects (such as where are 404 on the server requests) to marketing in order to get more insights about users of website

- 3 important changes for this shift from IT to marketing:
- 1.  JS enables to bypass IT and track data directy without the need to gather log files. 1996: Urchin released service called Quantified
- 2.  With Googles ad system, shift to information about success of campaign and referrer sites
- 3. New cost model for analytics service: pay for traffic you measure. Analytics as percentage of web revenues

- Cookies to identify unique users
- Page tagging: not only track technical data but also business context, e.g. analyst can say if user buys shoes he also buys jackets
- "Today, web analytics is a marketing discipline used to measure the effectiveness of communications strategies" p.83
- p. 84 visual history of web analytics
- More data: performance data, visitor opinion, usability data


% 2009 Jansen ch. 3. only provides a literature review over log analysis


% 2014 Singal
- 1990 birth of WWW
- Hit: when visitor requests html file. is in log file
- WebTrends 1993
- 1995: Analog. First free log file analysis software
- 1996: WebCounter: hit counter service
- JS page tagging
- 2003: Foundation of Web Analytics Association WAA by Edwards, Eisenberg and Sterne.
- 2006 In-Page analytics
- 2012 WAA renamed to Digital Analytics Association as all digital users are inlcuded in web analytics
- Image: Genesis of Web Analytics (nice graphic) with WAA, DAA



% 2015 Zheng:
- Ever growing since start of WWW
- From http traffic logging to use data tracking, analysis and reporting
- Log file analysis
- First browser Mosaic 1993
- WebTrends 1993
- 1995 Analog
- 1996 WebSideStory
- Page tagging
- Trends:
- mobile analytics,
- application specific analytics
- from web to digital analytics: understand entire digital footprint of users (see name change of WAA)


% 2015 Bekavac not really infos about history...
% 2019 Hussaina no history...

% 2019 Kumar
- all 500 imortant companies run websites
- Rise in the volume of data and internet users
- 


\subsection{Web Analytics Process}

% 2009 Jansen ch. 6.2
% 2009 Waisberg
% 2012 Kumar

% 2015 Bekavac:
% - Copies from Waisberg and Kaushik


% 2019 Hussaina
% 2019 Kumar


\paragraph{Data Sources: Log Analysis versus Page tagging}


% 2009 Croll
- Page tagging illustration p. 82


% 2009 Waisberg

% Hassler ch. 2

% Wolle Draft

% 2011 Marek

% 2011 Nakatani: Pros and cons

% 2014 Singal provides table with pros and cons

% 2015 Zheng

% 2015 Bekavac

% 2019 Kumar





% KPIs: ?
% -----

% 2014 Singal describes KPIs for different businesses

% 2015 Bekavac:
% - KPI definition as part of web analytics process
% - Table with KPIs related to business model




\subsection{Web Performance}
TODO move this to end of web analytics chapter? like again focus on performance...
\begin{itemize}
\item What is web performance? Why it matters
\item Overview of factors that impact performance, bottlenecks
\item Overview of measurement methods, techniques and metrics
\end{itemize}



% Bottlenecks of performance, big picture:
% 2013 Grigorik: Latency as bottleneck and not bandwith
% 2016 Witt




\subsection{Tools}

% TODO: move this to related work ?

\begin{itemize}
\item Some short overview about existing tools
\item Conclude that I use WPT for synthetic performance testing and GA for RUM
\end{itemize}

% Croll 2009 Choosing a analytics platform p. 144

% 2011 Marek: Choose a program



% 2011 Nakatani:
% "Web analytics tools collect click-stream data, track users navigation paths, process and present the data as meaningful information.
% - Categorizations:
% 1: By 4 different data collection methods
% 2: SaaS vs in-house
% 3: mobile vs non-mobile
% 4: Time lag



% 2014 Singal table history of analytics tools



% 2015 Bekavac
% - 5 categories of tools
% - Process of selecting tool
% - Table with features of tools


% 2016 Bartuskova: 1.2 Services for Website Speed Testing
% 2016 Kaur: Tools for Measuring the Performance of Websites
% 2019 Hussaina
% 2019 Kumar

% 2020 Quintel: Matomo
% Lighthouse Performance Scoring

% 2020 Heinemann 4.1.4




% ------------------------------------------------------------------------------------
% ------------------------------------------------------------------------------------
% ------------------------------------------------------------------------------------
% ------------------------------------------------------------------------------------




\section{Research Question}

\begin{itemize}
\item Difficulty of defining scope
\item Measuring performance of a web site impacts its performance or other effects take place / Observer effect
\item Why the research question is relevant
\end{itemize}

What is the research question of this thesis?
What is the goal?

% Tools: Kessler 2016 p. 576

% Measuring Real User Performance in the Browser: Avoiding the Observer Effect























