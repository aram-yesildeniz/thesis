\chapter{Introduction}

\begin{itemize}
	\item Explain structure and main goal of this thesis
	\item Introduction Chapter: Describe shortly all sections from this chapter and what the reader can expect
	\item Give short outlook to following chapter
\end{itemize}


% -----------------------------------
% -----------------------------------


\section{E-Commerce}


% [Internet]
\subsection{The Internet}

In the last 50 years, a new technology emerged, spread over the entire world and influenced many aspect of most peoples life.
Within the turmoil of the cold war, the United State's Advanced Research Projects Agency (ARPA) established in 1957 a communication network to bring together universities and their researches all around the country in order to be able compete against the USSR.
% cite 2011 COhen
What started as a tool for scientific collaboration evolved a half century later into the internet, a global network and phenomenon, to which every user with a dedicated device has access and can contribute to.
The internet is an integral part, if not the backbone of today's everyday life.
Users of the internet use it for everything, that is sending emails, watching television, chatting with friends, 
order lunch, checking the weather for the next day or renting motorized scooters.

% [Numbers]

In 2021, the internet has 4.66 billion users, which is around 60\% of the world population.\footnote{Following statistics are taken from \url{https://datareportal.com/reports/digital-2021-germany} [14.05.2021]}
Compared to 2020, the number of internet users increased by 7.3\%.
In Europe, more than 90\% of the population are also internet users.
For a developed country like Germany, the numbers are even more impressing:
94\% of the German population are using the internet with an average daily time of over 5h.

Those numbers demonstrate impressive that the internet is an integral part of our daily life.
Along the rise of internet users, transactions and processes falling under the term of e-commerce rise.
Before discussing the term "e-commerce" and take a grasp at its history and types, some statistics are presented to demonstrate the importance of e-commerce.


% [E-Commerce]
\subsection{E-Commerce}

\subsubsection{Introduction}

From the global data report, we can see that over 90\% of the world population visited an online retail site.
Over 76\% of the world population purchased a product online.
For most categories growth is over 15\%.
Again, for a western country like Germany, the numbers are higher:
92.5\% of the german population visited an online retail site and over 80\% purchased a product online.
And the usage is growing: growth of amount spent in category food and personal care is 28.6\%, and 17.6\% for fashion and beauty.

Revenue in e-commerce is constantly growing over the last 20 years, topping to 57.8 billion in 2019.\footnote{\url{https://einzelhandel.de/presse/zahlenfaktengrafiken/861-online-handel/1889-e-commerce-umsaetze} [14.05.2021]}

% [Corona: Even more growth]

The COVID-19 pandemic with its implications had and still has an not negligible impact on the growth of e-commerce.
Several measures were taken to stop the spread of the virus and the number of deaths, one of which was to minimize physical interaction between people.
This leads consequently to a shift of human interactions to the internet.
Along this, e-commerce benefits.
Bhatti et al concludes that "e-commerce enhanced by COVID-19". %cite 2020 bhatti

% [E-Commerce History]
\subsubsection{Short History}

E-Commerce, or electronic commerce, is according to the \textit{Encyclopædia Britannica} about "maintaining relationships and conducting business transactions that include selling information, services, and goods by means of computer telecommunications networks."\footnote{\url{https://www.britannica.com/technology/e-commerce} [19.05.2021]}
In short, e-commerce is about buying and selling products and services via the internet.

% cite hermogeno and check for Tian when paper available:
The success of e-commerce is tightly coupled to the vast advances of internet technology in the past years, like for example the development of the Electronic Data Interchange (EDI) starting from the 1960s, which standardised the communication between two machines.

Personal computers in 1980s and one of the first examples of online shopping is from CompuServe who introduced Electronic Mall in 1984.

Another milestone is Word Wide Web, introduced in 1990 by Tim Berners-Lee, made internet accessible to everyone.
First browser by Tim.

Social media since 2000 again offeres new possibilities for businesses and consumers alike to participate in e-commerce, e.g. for marketing, selling channels

New devices such as smart phones and tablets again decreases the hurdle to participate in e-commerce.
While in the time dimension e-commerce was already available all the time, with the new mobility its also available everywhere.
More flexible.

With the ongoing progress in technology, also e-commerce can expect a shining future with trends such as AI recommendation systems, outstanding UX thanks to Virtual Reality, simple payment methods with crypto etc.\footnote{\url{https://www.spiralytics.com/blog/past-present-future-ecommerce/} [19.05.2021]}


% [Types]
\subsubsection{Types}

Multiple types in e-commerce are existing. They emerge from the possible combinations between the actors \textit{business}, \textit{consumer} and \textit{government}. %cite dos santos

\begin{itemize}
\item B2B: Business to Business
\item B2C: Business to Consumer
\item C2C: Consumer to Consumer
\item G2C: Government to Consumer
\item B2G: Business to Government
\item G2G: Government to Government
\end{itemize}

% [B2C]
\paragraph{B2C}
Business to Consumer in e-commerce describes basically online shopping, by means of a business offering its services and products to the consumer over the World Wide Web.
The consumer can browse within an online shop through the presented products and services and order them directly via the website.
A variety of payment and delivery options conclude the B2C type.
% cite Heinemann p. 75

For a aspiring business, multiple ready to use software solutions to install an online shop are existing, for example. Shopify, ePages, Magento or WooCommerce. % cite Handbuch Online-Shop

% [Amazon]
A famous example of a B2C company is Amazon.
On the 16th of July in 1995, Amazon launched as a website and entered the stock market on the 15th of May 1997.%cite Der Allesverkäufer p. 47
Amazon is successful, the stock started with a price of X, which is at the time of this writing at Y.\footnote{\url{https://finance.yahoo.com/quote/AMZN?p=AMZN} [19.05.2021]}
Today Amazon employs over 1 million employees\footnote{\url{https://www.statista.com/statistics/234488/number-of-amazon-employees/} [19.05.2021]} and serves the wishes of 200 million paying prime members.\footnote{\url{https://www.statista.com/statistics/829113/number-of-paying-amazon-prime-members/} [20.05.2021]}


% [Pro and Con]
By taking a quick look at the pros and cons of maintaining an online shop, we can see that some of the advantages are that there is no need of a real house to present and sell the products, the virtual shop is available to the consumer at any time and has to closing hours; there is a high potential for the online shop as it is part of growing market; online business is scalable; due to tracking algorithms precise targeting as well as data analysis is possible; to start with an online business, not that much float is needed and there are in general lower costs; it is possible to provide a personalized customer experience; 

Some disadvantages are that the speed of market is rapid, competitors arise everyday everywhere, technology evolves quickly while consumers expectations go high.
% cite 2019 hermogeno and % cite erfolgreicher online handel for dummies


% [Performance of online shop is important]
Another downside is that there is no direct or physical concat with the consumer.
As described above, online shopping takes place in the World Wide Web domain.
Consequently, in person interaction between a buyer and a seller is not possible and the shopping event takes place on a website.
Deriving from this fact, the overall virtual user experience needs to be outstanding in order to stay ahead of competition.
The performance of the online shop is one part of the user experience.

In the next section, I will describe the findings between the correlation between user satisfaction and the performance of the retailers web presence.



% ------------------------------------------


\subsection{User Satisfaction and Performance}

% [User Satisfaction]
The aim of this thesis is not to deep dive into terms and concepts or the non-trivial problem of defining user satisfaction, usability or the like.
Therefore the term user satisfaction is in this context loosely defined as how happy the user is with the website he or she interacts with.\footnote{For a discussion cf. "User satisfaction measurement" in } %cite 2010 Islam

% [Performance]
For this context, performance can be understood as the the speed of an online shop, e.g. how long it takes the page to load, how quickly the user can interact with the page, and how the user perceives the performance of the website.
Later we will see that measuring performance is not that trivial and a lot of ideas and metrics are existing to measure it.

% [SpeedHub]
% QUESTION: should i cite only the keynote or the literature used in the keynote directly ?
\subsubsection{SpeedHub}
A plethora of information and studies about the phenomenon of user satisfaction and web site performance is collected at \textit{SpeedHub.org}, a portal by \textit{Baqend} in cooperation with \textit{Google} which provides "the largest systematic study of Mobile Site Speed and the Impact on E-Commerce."\footnote{\url{https://www.speedhub.org/} [21.05.2021]}
On the hub, not only studies and reports are available, but also collections of videos and blog posts.

% [code.talks 2019]
In his presentation at code talks 2019, Felix Gessert summarizes the findings and provides insights to the most relevant aspects and questions of the study: %cite 2019 Gessert

% [User Profile and Psychology]
The first observation when tackling the question regarding a correlation between the performance of a system and the user satisfaction, is that the users have to be distinguished which leads to the concept of a \textit{User Profile}: Regarding gender, young woman are the most demanding consumers and buy less on slow pages.
Generally, people between 18 and 24 have higher expectations regarding site speed than their older counterparts.
There are also differences between nations and regions, for example people from Japan have the highest expectations, which for a certainty coheres with the technological progress in this country.
Not only the expectations themselves differ geographically, but also how speed influences the users, for example "speed influences New Yorkers more than Californians."

What all users have in common is their human psychology. With respect to performance, researchers generally suggest to keep waiting times under 1000 ms in order to keep the users attention.

% [Devices]
After considering the user itself, the next step is to investigate the influence of the device in use: Studies show that mobile users are more likely to buy products and services than their colleagues using a desktop computer, where iOS users have generally more expectations regarding site speed.
 
% [Context]
Last but not least is the context and the users condition important, where naturally relaxed and calm users perceive sites faster than stressed users or people that are in a hurry.
Also when on the move, users experience sites slower.

% [Studies]
There are many real world examples and studies existing which prove and demonstrate the importance of site speed with respect to user satisfaction and eventually revenue:
\textit{Amazon} fount out that a decrease of 100 ms in page loading leads to -1\% conversion rates.
If the site loads 100 ms faster, \textit{Walmart} observed that the revenue increases by 1\%.
For \textit{Zalando}, increasing site speed by 100 ms leads to an uplift of 0.7\% revenue per session.

% [SEO]
Search Engine Optimization is heavily impacted by load speed:
For \textit{Google}, 500 ms slower sites lead to a decrease of 20\% in traffic.
\textit{GQs} traffic increased by 80\% after the page load went down from 7 s to 2 s.
And for \textit{Pinterest}, 40\% faster loads led to 15\% more SEO traffic.

% [Engagement & Satisfaction]
Also the user engagement and satisfaction rely heavily on load times:  \textit{Forrester} noted an increase of 60\% for the session length while brining down the load time 80\%.
\textit{Akamai} monitored that the bounce rate climbed up incredible 103\% when the load time increased by 2 s.
And for the \textit{AberdeedGroup}, the customer satisfaction dropped by 16\% at one more second delay in response times.

To summarize, many studies and real world examples prove and demonstrate that faster web sites and online shops cause a better user experience and typically lead to happier customers. 
Concluding in commercial terms, one can say with certainty that page speed equals money.
\\

A scientific method is needed in order to test the impact of performance on the users.
One of them is A/B Testing, which is the content of the next section.
After a discussion of A/B Testing, I will move on to the examination of Web Analytics, a term which subsumes methods, tools and instruments for businesses to better understand their business and customers.


\subsubsection{A/B Testing}

% How to quantify, measure impact of performance?
% Try to be scientifically solid

% 2012 Kessler 17.2  AB Tests



% 2016 Kohavi

% 2017 Kohavi
% 2018 Morys
% 2018 Fabijan
% 2020 Heinemann 4.1.4





% ------------------------------------------------------------------------------------
% ------------------------------------------------------------------------------------



\section{Web Analytics}

\begin{itemize}
\item Historical background and contextualisation, usage, definition
\item Web Analytics Process
\item Mechanisms, Measurement methods / Collecting data: Log file analysis, client site page tagging, alternatives
\item KPIs ?
\end{itemize}


\subsection{Introduction}

% Definitions:
% -----------


% 2011 Nakatani:
% Web analytics is used to understand online customers and their behaviors, design actions influential to them, and ultimately foster behaviors beneficial to the business and achieve the organization’s goal.


% 2014 Singal: "Web Analytics is the objective tracking, collection, measurement, reporting and analysis of quantitative internet data to optimize websites and web marketing initiatives."

% 2015 Bekavac:

% "According to the official definition of [13], web analytics refers to a combination of (a) measuring, (b) acquisition, (c) analyzing and (d) reporting of data collected from the Internet with the aim of understanding and optimizing web experience."

% "the analysis of qualitative and quantitative data on the website in order to continuously improve the online experience of visitors, which leads to more efficient and effective realization of the company’s planned goals"




\subsection{Short History}


% 2009 Croll p. 69
% 2009 Jansen ch. 3

% 2014 Singal
% - Genesis of Web Analytics (nice graphic) with WAA, DAA
% - History of tools

% 2015 Zheng
% 2015 Bekavac
% 2019 Hussaina
% 2019 Kumar


\subsection{Web Analytics Process}
% Process:
% --------


% 2009 Jansen ch. 6.2
% 2009 Waisberg
% 2012 Kumar

% 2015 Bekavac:
% - Copies from Waisberg and Kaushik


% 2019 Hussaina
% 2019 Kumar


% Log vs Page tagging:
% --------------------


% 2009 Waisberg

% Hassler ch. 2

% Wolle Draft

% 2011 Marek


% 2011 Nakatani: Pros and cons


% 2014 Singal provides table with pros and cons

% 2015 Bekavac

% 2019 Kumar



% KPIs:
% -----

% 2014 Singal describes KPIs for different businesses

% 2015 Bekavac:
% - KPI definition as part of web analytics process
% - Table with KPIs related to business model


\subsection{Web Performance}

\begin{itemize}
\item What is web performance? Why it matters
\item Overview of factors that impact performance, bottlenecks
\item Overview of measurement methods, techniques and metrics
\end{itemize}



% Bottlenecks of performance, big picture:
% 2013 Grigorik: Latency as bottleneck and not bandwith
% 2016 Witt




\subsection{Tools}

% TODO: move this to related work ?

\begin{itemize}
\item Some short overview about existing tools
\item Conclude that I use WPT for synthetic performance testing and GA for RUM
\end{itemize}

% 2011 Marek: Choose a program



% 2011 Nakatani:
% "Web analytics tools collect click-stream data, track users navigation paths, process and present the data as meaningful information.
% - Categorizations:
% 1: By 4 different data collection methods
% 2: SaaS vs in-house
% 3: mobile vs non-mobile
% 4: Time lag



% 2015 Bekavac
% - 5 categories of tools
% - Process of selecting tool
% - Table with features of tools


% 2016 Bartuskova: 1.2 Services for Website Speed Testing
% 2016 Kaur: Tools for Measuring the Performance of Websites
% 2019 Hussaina
% 2019 Kumar

% 2020 Quintel: Matomo
% Lighthouse Performance Scoring

% 2020 Heinemann 4.1.4




% -----------------------------------
% -----------------------------------



\section{Research Question}

\begin{itemize}
\item Difficulty of defining scope
\item Measuring performance of a web site impacts its performance or other effects take place / Observer effect
\item Why the research question is relevant
\end{itemize}


% Tools: Kessler 2016 p. 576

% Measuring Real User Performance in the Browser: Avoiding the Observer Effect























