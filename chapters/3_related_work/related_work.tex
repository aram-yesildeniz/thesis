\chapter{Related Work}

\begin{itemize}
	\item Last chapter...
	\item This chapter: Describe shortly all sections from this chapter
	\item In the next chapter...
\end{itemize}

\begin{itemize}
\item This chapter should list research which covers and explores questions relevant for this thesis, such as:
	\begin{itemize}
	\item Metrics: New metrics, meaning of metrics, difficulties of defining metrics, etc.
	\item Overview, evaluation and comparison of measurement tools and methods
	\item If available: Impact of RUM on performance
	\end{itemize}
\end{itemize}


% ---------------------------------------------


% 2016 Viscomi
% https://docs.webpagetest.org/

\section{WebPageTest}

\begin{itemize}
\item Overview
\item Configuration
\item Private Instances
\end{itemize}




\subsection{Overview}

\begin{itemize}
\item What is it
\item Why to use it, Who uses it, how to use it
\item Waterfall and Grades
\end{itemize}


\subsection{Configuration}

\begin{itemize}
\item Caching, repeat view
\item Traffic shaping
\end{itemize}


\subsection{Private Instances}

\begin{itemize}
\item Architecture
\item AWS
\item Docker localhost
\item Bulk tests
\end{itemize}




% ---------------------------------------------


\section{Google Analytics}

\begin{itemize}
    \item Custom metrics with Google Web Vitals as example
    \item Show how to include GA script (analytics.js, gtag, Tag Manager, etc.)
    \item Show some real life examples how script code is included into page, e.g. from Amazon, Otto etc
\end{itemize}

% Kessler 2012 p. 581
% marek 2011: glossary




% ---------------------------------------------
% ---------------------------------------------



\section{Research}


\begin{itemize}
\item Research exists about topics like: ....
\item Here i will provide a list of in my eyes relevant papers, summaries them and discuss why this is important for my research
\end{itemize}

% TODO: Think about categorising papers


\subsection{some title for first category}

% 2014 Singal: WEB ANALYTICS: STATE-OF-ART & LITERATURE ASSESSMENT
\paragraph{2014 Singal}

I.
- Describes history of web analytics and tools
- Provides definitions and taxonomy for metrics
- Describes log file vs page tagging
- Describes KPIs

II.
- Lit. overview for KPIs and Web Metrics
- Lit. overview for "Trust"
- Lit. overview for "Fuzzy"
-> What are does categories?


III.
- Some other literature worth mentioning


IV.
- Describes 8 open challenges for researchers


% ---------------------------------------------


\paragraph{2015 Bekavac}

- Two parts:
	- 1: Some general overview of web analytics, tools and metrics, KPIs etc
	- 2: Empirical study about employees satisfaction of used web analytics tools

1: 
- 9 web business models and 5 common goals
- Hypothesis: Web analytics tools track and improve a user’s satisfaction with web-based business models.
- Web anayltics defintion. Log files vs Site Tagging
- Web Analytics process
- Tools: 5 categories, Process of selecting tool, Table with features of different tools
- Web metrics categories, Table with business models and their KPIs

2:
- Which tools are used for which purpose / Activity
- Users satisfaction



% ---------------------------------------------


\subsection{Research about Tools}

% 2011 Nakatani: A web analytics tool selection method: an analytical hierarchy process approach
\paragraph{2011 Nakatani}

- Gives some arguments why web analytics is important for business
- Provides different categorizations for web analytics tools
- Gives pros and cons of log file analysis and page tagging
- Provides tool selection method based on AHP (Analytic Hierarchy Process)



% ---------------------------------------------

% https://www.kaushik.net/avinash/web-analytics-tool-selection-three-questions-to-ask-yourself/
\paragraph{Kaushik 2007}
- Provides 3 questions which help to choose web analytics tools


% ---------------------------------------------


% 2013 Wang: WProf: Profiler / new tool for finding bottlenecks





% ---------------------------------------------


% 2016 Kaur: Tool comparison: This paper evaluated the university websites of Punjab using four automated tools and gives the comparative results of various factors using these tools.







\subsection{Research about Metrics}
% 2018 Netravali: Ready Index: new metric
% 2019 Enghard Pitfalls







- Dont know:
% 2015  Cito: IDENTIFYING WEB PERFORMANCE DEGRADATIONS THROUGH SYNTHETIC AND REAL-USER MONITORING

% 2016 Bartuskova: Our results indicate that the choice of service and location affects significantly results of website speed testing.






















